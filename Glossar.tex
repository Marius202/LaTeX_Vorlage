% very useful: https://tex.stackexchange.com/questions/8946/how-to-combine-acronym-and-glossary
% put "\hfill \\" at the beginnig of a description to start the definition at a new line

\makeglossaries
\renewcommand{\glossaryentrynumbers}[1]{(#1)}

%%%%%%%%%%%%%%%%%%%%%%%%%%%%%%%%%%%%%%%%%%%%%%%%%%%%%%%%%%%%%%%%%%%%%%%%%%%%%%%%%%%%%%%%%%%%%%%%%%%%
%                                                                                                  %
%                                             ACRONYMS                                             %
%                                                                                                  %
%%%%%%%%%%%%%%%%%%%%%%%%%%%%%%%%%%%%%%%%%%%%%%%%%%%%%%%%%%%%%%%%%%%%%%%%%%%%%%%%%%%%%%%%%%%%%%%%%%%%

%---------------------------------------------------------------------------------------------------
% Computer-aided design (CAD)
%---------------------------------------------------------------------------------------------------
\newglossaryentry{cad}{
	type=\acronymtype,
	name={CAD},
	description={\textbf{C}omputer-\textbf{a}ided \textbf{d}esign},
	first={Computer-aided design (CAD)}
}

%%%%%%%%%%%%%%%%%%%%%%%%%%%%%%%%%%%%%%%%%%%%%%%%%%%%%%%%%%%%%%%%%%%%%%%%%%%%%%%%%%%%%%%%%%%%%%%%%%%%
%                                                                                                  %
%                                     ACRONYMS WITH GLOSSARIES                                     %
%                                                                                                  %
%%%%%%%%%%%%%%%%%%%%%%%%%%%%%%%%%%%%%%%%%%%%%%%%%%%%%%%%%%%%%%%%%%%%%%%%%%%%%%%%%%%%%%%%%%%%%%%%%%%%

%---------------------------------------------------------------------------------------------------
% Human Machine Interface (HMI)
%---------------------------------------------------------------------------------------------------
\newglossaryentry{glos:hmi}{
	name={HMI},
	description={Ein HMI, \textbf{H}uman-\textbf{M}achine \textbf{I}nterface (dt. Mensch-Maschine Schnittstelle), ist eine Schnittstelle zwischen dem Benutzer und einer Maschine und ermöglicht die Kommunikation mit dem Gerät. Es kann sich dabei um Hard- oder Software handeln. Im Rahmen dieses Berichts bezieht sich die {\dq}Zimmer-HMI{\dq} auf eine Software, mit der Daten von Greifern gelesen werden sowie Greifer gesteuert werden können}
}

\newglossaryentry{hmi}{
	type=\acronymtype,
	name={HMI\glsadd{glos:hmi}},
	description={\textbf{H}uman-\textbf{M}achine \textbf{I}nterface},
	first={Human-Machine Interface (HMI)\glsadd{glos:hmi}},
	see=[Glossar:]{glos:hmi}
}

%%%%%%%%%%%%%%%%%%%%%%%%%%%%%%%%%%%%%%%%%%%%%%%%%%%%%%%%%%%%%%%%%%%%%%%%%%%%%%%%%%%%%%%%%%%%%%%%%%%%
%                                                                                                  %
%                                            GLOSSARIES                                            %
%                                                                                                  %
%%%%%%%%%%%%%%%%%%%%%%%%%%%%%%%%%%%%%%%%%%%%%%%%%%%%%%%%%%%%%%%%%%%%%%%%%%%%%%%%%%%%%%%%%%%%%%%%%%%%

%--------------------------------------------------------------------------------------------------
% LuaLaTeX
%--------------------------------------------------------------------------------------------------
\newglossaryentry{glos:lualatex}{
	name={Lua\LaTeX},
	description={Lua{\LaTeX} basiert auf Lua{\TeX}...}
}
